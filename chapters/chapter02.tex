\documentclass[../thesis.tex]{subfiles}

\begin{document}

Triết lý giáo dục của Việt Nam Cộng hòa là nhân bản, dân tộc, và khai phóng.[14]
Năm 1958, dưới thời Bộ trưởng Bộ Quốc gia Giáo dục Trần Hữu Thế, Việt Nam Cộng hòa nhóm họp Đại hội Giáo dục Quốc gia (lần I) tại Sài Gòn. Đại hội này quy tụ nhiều phụ huynh học sinh, thân hào nhân sĩ, học giả, đại diện của quân đội, chính quyền và các tổ chức quần chúng, đại diện ngành văn hóa và giáo dục các cấp từ tiểu học đến đại học, từ phổ thông đến kỹ thuật... Ba nguyên tắc "nhân bản", "dân tộc", và "khai phóng" được chính thức hóa ở hội nghị này.[15][16] Đây là những nguyên tắc làm nền tảng cho triết lý giáo dục của Việt Nam Cộng hòa, được ghi cụ thể trong tài liệu Những nguyên tắc căn bản do Bộ Quốc gia Giáo dục ấn hành năm 1959 và sau đó trong Hiến pháp Việt Nam Cộng hòa (1967). Theo Nguyễn Thanh Liêm trong một công trình nghiên cứu xuất bản tại Santa Ana năm 2006, và theo tài liệu Chính sách văn hóa giáo dục của Hội đồng Văn hóa Giáo dục Việt Nam Cộng hòa ấn hành năm 1972, những nguyên tắc trên có thể hiểu như sau[17]:
Nhân bản. Triết lý nhân bản chủ trương con người có địa vị quan trọng trong thế gian này; lấy con người làm gốc, lấy cuộc sống của con người trong cuộc đời này làm căn bản; xem con người như một cứu cánh chứ không phải như một phương tiện hay công cụ phục vụ cho mục tiêu của bất cứ cá nhân, đảng phái, hay tổ chức nào khác. Đề cao những giá trị thiêng liêng của con người. Chủ trương sự phát triển quân bình và toàn diện của mỗi người và mọi người. Triết lý nhân bản chấp nhận có sự khác biệt giữa các cá nhân, nhưng không chấp nhận việc sử dụng sự khác biệt đó để đánh giá con người, và không chấp nhận sự kỳ thị hay phân biệt giàu nghèo, địa phương, tôn giáo, chủng tộc... Với triết lý nhân bản, mọi người có giá trị như nhau và đều có quyền được hưởng những cơ hội đồng đều về giáo dục.[14][17]
Dân tộc. Giáo dục tôn trọng giá trị đặc thù, các truyền thống tốt đẹp của dân tộc trong mọi sinh hoạt liên hệ tới gia đình, nghề nghiệp, và quốc gia. Giáo dục phải biểu hiện, bảo tồn và phát huy được những tinh hoa hay những truyền thống tốt đẹp của văn hóa dân tộc để không bị mất đi hay tan biến trong những nền văn hóa khác. Giáo dục còn nhằm bảo đảm sự đoàn kết và trường tồn của dân tộc, sự phát triển điều hòa và toàn diện của quốc gia.[14][17]
Khai phóng. Tinh thần dân tộc không nhất thiết phải bảo thủ, không nhất thiết phải đóng cửa. Ngược lại, giáo dục phải không ngừng hướng tới sự tiến bộ, tôn trọng tinh thần khoa học, mở rộng tiếp nhận những kiến thức khoa học kỹ thuật tân tiến trên thế giới, tiếp nhận tinh thần dân chủ, phát triển xã hội, tinh hoa văn hóa nhân loại để góp phần vào việc hiện đại hóa quốc gia và xã hội, làm cho xã hội tiến bộ tiếp cận với văn minh thế giới, góp phần phát triển sự cảm thông và hợp tác quốc tế, tích cực đóng góp vào sự thăng tiến nhân loại.[14][17]
Từ những nguyên tắc căn bản ở trên, chính quyền Việt Nam Cộng hòa đề ra những mục tiêu chính cho nền giáo dục của mình. Những mục tiêu này được đề ra là để nhằm trả lời cho câu hỏi: sau khi nhận được sự giáo dục, những người đi học sẽ trở nên người như thế nào đối với cá nhân mình, đối với gia đình, quốc gia, xã hội, và nhân loại:
Phát triển toàn diện mỗi cá nhân. Trong tinh thần tôn trọng nhân cách và giá trị của cá nhân học sinh, giáo dục hướng vào việc phát triển toàn diện mỗi cá nhân theo bản tính tự nhiên của mỗi người và theo những quy luật phát triển tự nhiên cả về thể chất lẫn tâm lý. Nhân cách và khả năng riêng của học sinh được lưu ý đúng mức. Cung cấp cho học sinh đầy đủ thông tin và dữ kiện để học sinh phán đoán, lựa chọn; không che giấu thông tin hay chỉ cung cấp những thông tin chọn lọc thiếu trung thực theo một chủ trương, hướng đi định sẵn nào.\footnote{Trần Văn Lục. Một thời để nhớ: những sự thật về cố Tổng thống Ngô Đình Diệm và nền Đệ Nhất Cộng hòa. Westminster, CA: Nguyệt san Diễn đàn Giáo dân, 2011. tr. 265–266.}
Phát triển tinh thần quốc gia ở mỗi học sinh. Điều này thực hiện bằng cách: giúp học sinh hiểu biết hoàn cảnh xã hội, môi trường sống, và lối sống của người dân; giúp học sinh hiểu biết lịch sử nước nhà, yêu thương xứ sở mình, ca ngợi tinh thần đoàn kết, tranh đấu của người dân trong việc chống ngoại xâm bảo vệ tổ quốc; giúp học sinh học tiếng Việt và sử dụng tiếng Việt một cách có hiệu quả; giúp học sinh nhận biết nét đẹp của quê hương xứ sở, những tài nguyên phong phú của quốc gia, những phẩm hạnh truyền thống của dân tộc; giúp học sinh bảo tồn những truyền thống tốt đẹp, những phong tục giá trị của quốc gia; giúp học sinh có tinh thần tự tin, tự lực, và tự lập.[14]
Phát triển tinh thần dân chủ và tinh thần khoa học. Điều này thực hiện bằng cách: giúp học sinh tổ chức những nhóm làm việc độc lập qua đó phát triển tinh thần cộng đồng và ý thức tập thể; giúp học sinh phát triển óc phán đoán với tinh thần trách nhiệm và kỷ luật; giúp phát triển tính tò mò và tinh thần khoa học; giúp học sinh có khả năng tiếp nhận những giá trị văn hóa của nhân loại.[14]
Tuy nhiên trong giai đoạn 1955-1963, dưới thời Ngô Đình Diệm, nền giáo dục Việt Nam Cộng hòa bị xem là thiên vị Thiên Chúa giáo nặng nề. Ngô Đình Diệm dành cho Giáo hội Thiên Chúa giáo quyền chi phối các trường (kể cả các trường không phải là của giáo hội) về mặt tinh thần, cốt bảo đảm thực hiện được nội dung giáo dục “Duy linh” mà thực chất là nội dung thần học theo lối triết học kinh viện thời Trung cổ. Phần lớn các học bổng đi học nước ngoài đều rơi vào tay các linh mục hoặc sinh viên Thiên Chúa giáo.

Do chính sách ưu đãi tài chính cùng với những đặc quyền mà Ngô Đình Diệm dành cho, hệ thống trường tư thục của Thiên Chúa giáo phát triển rất nhanh. Avro Manhattan thống kê rằng: Từ năm 1953 đến năm 1963, khắp miền Nam đã xây dựng 145 trường cấp II và cấp III, riêng ở Sài Gòn có 30 trường với tổng số 62.324 học sinh. Cũng trong cùng thời gian này, Giáo hội Thiên Chúa giáo ở miền Nam Việt Nam, từ chỗ chỉ có 3 trường cấp II và III trong năm 1953, đến năm 1963 đã lên tới 1.060 trường[24]. Có nơi Linh mục dùng uy thế của mình để phụ huynh không cho con học trường công mà phải vào học trường của Giáo hội, nên trường tư thục của Giáo hội làm tê liệt cả trường công vì không tuyển được học sinh[25]
Sau khi Ngô Đình Diệm bị lật đổ, các chính sách thiên vị Thiên Chúa giáo trong nền giáo dục mới kết thúc, tuy nhiên những nội dung liên quan đến chủ nghĩa cộng sản và việc Việt Minh lãnh đạo chiến tranh chống Pháp thì vẫn bị cấm giảng dạy.


\end{document}